\documentclass[a4paper,11pt]{article}


% define the title
\author{Xue, Yuan}
\title{Econometric II Notes}


\usepackage{verbatim}
\usepackage{amsmath}
\usepackage{amssymb}
\usepackage{tikz}
\usetikzlibrary{graphs}

%\pagestyle{headings}
%\thispagestyle{headings}

\begin{document}
\maketitle
\tableofcontents
\section{Philosophy}
Learn how to read papers doing econometrics. Learn approaches to empirical research.
\subsection{External validity and internal validity}
\begin{itemize}
	\item[-] Internal validity
	
The degree to which the research design yields estimation that reflects the true causal effect for the population, location or time under consideration.
	\item[-] External validity
	
The degree to which the research design yields estimation that can be used to predict the causal effect in other population, location or time.
	\item[-] Threats to internal validity
	\begin{itemize}
		\item[-] Selection bias, omitted variable bias, endogeneity.
		\item[-] Misspecification of function form (linearity, interactions).
		\item[-] Attrition
		\item[-] Measurement error
	\end{itemize}
	\item[-] Threats to external validity
	\begin{itemize}
		\item[-] Interaction of the setting (agents, institutions, time) and treatment that is relevant.
		\item[-] Internal validity
	\end{itemize}
\end{itemize}
\paragraph{Model for individual $i$, group $g$, location $l$, time $t$}
\begin{equation}
	y_{igtl}=\alpha +(\beta_0+\beta_1x_{igtl})T_{igtl}+w_{igtl}\theta + \epsilon_{iglt}
\end{equation}
$x_{igtl}$: Interactive confounders.\\
$T_{igtl}$: Treatment Effect.\\
$T_{igtl}$: Additive confounders.

An experiment in which $T_{iglt}$ is randomized will be internally valid. It will yield an estimation of effect for the distribution of $x$ represented in the study (from comparison of mean).
\paragraph{Whether the estimation is externally valid depends on:}
\begin{enumerate}
	\item The degree to which the population/location/time of interest is reflected in the study.
	\item The degree to which effects are heterogeneous.
	\item Internal validity.
\end{enumerate}
\subsection{Structural vs reduced form estimation}
\paragraph{``The battle ground"}
	\begin{itemize}
		\item[-] Credibility of theoretical vs statistical assumptions.
		\item[-] How to achieve external validity.
		\item[-] Types of questions that should be asked.		
	\end{itemize}
\paragraph{Structural approach}
	\begin{itemize}
		\item[-] Using economic theory as the starting point.
		\item[-] Modelling a system involving a collection of endogenous variables, exogenous variables\ldots (maximization of profits/utility)
		\item[-] Choice:
		\begin{itemize}
			\item[-] What to specify as endogenous or exgonenous variables are determined.
			\item[-] What function form to use.
			\item[$\rightarrow$] Generate equations corresponding to choices and/or aggregates.
			\item[-] ML estimation of non-linear simultaneous equations.
			\item[-] Parameters are ``fundamental", i.e., argubaly not context dependent.
			\item[-] Allows for generality.
			\item[-] Can simulate behavioral responses to policy as treatment happens.
			\item[-] Can conduct welfare calculations.
		\end{itemize}
		\item[-] Weakness:
		\begin{itemize}
			\item[-] Comparatively weak support for exogenous assessment.
			\item[-] Theoretical assumption may be overly restrictive.
		\end{itemize}
	\end{itemize}
\paragraph{Reduced form/design-based/quasi-experimental approach}
\begin{itemize}
	\item[-]Doesn't try to uncover ``the model"generating data.
	\item[-]Focuses on obtaining credible causal estimation of parameters of interest, not necessarily ``fundamental" parameters.
	\item[-]\underline{Example}: Document effects of class size reductions on test scores instead of modeling test scores (structural).
\end{itemize}

\begin{figure}[!h]
\centering
	\tikz {
		\node (a) at (0,0.5) {test scores};
		\node (b) at (3,1) {teacher effort};
		\node (c) at (3,0) {parent effort};
		\node (d) at (6,.5) {class size};
		\graph {
			(a) <- (b);
			(a) <- (c);
			(b) <-> (c);
			(b) <-> (d);
			(d) ->[bend left] (a);
		};
}
\end{figure}

\begin{figure}[!h]
\centering
	\tikz {
		\node (a) at (0,0) {test scores};
		\node (b) [draw] at (0,1) {black box};
		\node (c) at (0,2) {exogenous shock/change ($\Delta$) to class size};
		\graph {
			(a) <- (b);
			(b) <- (c);
		};
}
\end{figure}

\end{document}